\documentclass[a4paper,11pt]{article}

\usepackage[dutch]{babel}
\usepackage[utf8]{inputenc}
\usepackage[T1]{fontenc}
\usepackage[margin=2.54cm]{geometry}

% Package om kaders te tekenen rond tekst. Gebruik \begin{framed} en \end{framed} rond de tekst die je wil omkaderen.
\usepackage{framed}

\title{Draaiboek OOR}
\author{Overkoepelende Onderwijsraad \\ Scientica Leuven vzw (Scientica OOR) \\ Facultair Overlegorgaan Faculteit Wetenschappen}
% Gebruik date voor het versienummer. Voor een herziening van een versie, gebruik <nummer>.<revisienummer> (bijvoorbeeld 2020.1 voor een eerste herziening).
\date{Versie 2020}

\begin{document}
	\maketitle
	
	\vspace*{\fill}
	
	\begin{framed}
		Een minimale invulling van een aspect van de werking van OOR zal aangeduid worden met een kader, zoals dit. Hierdoor kan de minimale werking van OOR nagegaan worden zonder het volledige document in detail na te moeten lezen.
	\end{framed}

	% Voeg je naam toe wanneer je toevoegingen hebt gemaakt aan dit draaiboek.
	\textbf{Auteurs:} Ruben Van Laer
	
	\newpage
	
	\tableofcontents
	
	\newpage
	
	\section{Statuten Scientica Leuven vzw}
	
	\section{Huishoudelijk Reglement OOR}
	
	\subsection{Aanpassingen}
	
	\section{Logo en Embleem}
	
	\section{Website OOR}
	
	\section{E-mailadressen}
	
	\section{OOR-overleg}
	
	\subsection{Uitnodiging}
	
	\subsection{Agenda}
	
	\section{Verslagen}
	
	\subsection{Aanwezigheden}
	
	\subsection{Voorbereiding Stura AV}
	
	\section{OOR trakteert}
	
	\section{Scientica AV}
	
	\section{Faculteit Wetenschappen}
	
	\section{Studentenraad KU Leuven}
	\label{stura}
	
	\section{Campusbibliotheek Arenberg (CBA)}
	
	\section{Voormalige Bestuurders}
	
	
\end{document}