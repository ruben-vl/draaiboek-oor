\documentclass[a4paper,11pt]{article}

\usepackage[dutch]{babel}
\usepackage[utf8]{inputenc}
\usepackage[T1]{fontenc}
\usepackage[margin=2.54cm]{geometry}

% Package om kaders te tekenen rond tekst. Gebruik \begin{framed} en \end{framed} rond de tekst die je wil omkaderen.
\usepackage{framed}

\usepackage{hyperref}

\title{Draaiboek OOR}
\author{Overkoepelende Onderwijsraad \\ Scientica Leuven vzw (Scientica OOR) \\ Facultair Overlegorgaan Faculteit Wetenschappen}
% Gebruik date voor het versienummer. Voor een herziening van een versie, gebruik <nummer>.<revisienummer> (bijvoorbeeld 2020.1 voor een eerste herziening).
\date{Versie 2020}

\begin{document}
	\maketitle
	
	% Zorgt ervoor de de rest van de tekst onderaan de pagina staat.
	\vspace*{\fill}
	
	\textbf{Waarschuwing:} Dit document is niet absoluut. Het is slechts een beschrijving van de werking van OOR op dit moment en de ideeën die er circuleren over hoe OOR zou kunnen werken. Gebruik dit document dus enkel ter inspiratie of als documentatie, niet als een bron die strikt gevolgd moet worden. De voornaamste reden voor het maken van dit draaiboek is om een toekomstig onervaren bestuur de mogelijkheid te geven begeleid te worden, zonder voormalige bestuurders op te moeten sporen.\newline
	
	\textbf{Opmerkingen?} Pas gerust zaken in dit document zelf aan. Het is publiekelijk beschikbaar op \hyperlink{https://github.com/ruben-vl/draaiboek-oor}{github}\footnote{Indien je het document niet elektronisch hebt en dus niet op de link kan klikken, kan je de repository op volgende url vinden: https://github.com/ruben-vl/draaiboek-oor} en aanpassingen kunnen via een pull request in de master branch toegevoegd worden. Om je te laten toevoegen als collaborator of voor andere zaken kan je mailen naar rubenvanlaer@mail.com.
	
	\begin{framed}
		Een minimale invulling van een aspect van de werking van OOR zal aangeduid worden met een kader, zoals dit. Hierdoor kan de minimale werking van OOR nagegaan worden zonder het volledige document in detail na te moeten lezen.
	\end{framed}

	% Voeg je naam toe wanneer je toevoegingen hebt gemaakt aan dit draaiboek.
	\textbf{Auteurs:} Ruben Van Laer
	
	\newpage
	
	\tableofcontents
	
	\newpage
	
	\section{Statuten Scientica Leuven vzw}
	
	\section{Huishoudelijk Reglement OOR}
	
	\subsection{Aanpassingen}
	
	\section{Logo en Embleem}
	
	\section{Website OOR}
	
	\section{E-mailadressen}
	
	\section{OOR-overleg}
	
	\subsection{Uitnodiging}
	
	\subsection{Agenda}
	
	\section{Verslagen}
	
	\subsection{Aanwezigheden}
	
	\subsection{Voorbereiding Stura AV}
	
	\section{OOR trakteert}
	
	\section{Scientica AV}
	
	\section{Faculteit Wetenschappen}
	
	\section{Studentenraad KU Leuven}
	\label{stura}
	
	\section{Campusbibliotheek Arenberg (CBA)}
	
	\section{Voormalige Bestuurders}
	
	
\end{document}