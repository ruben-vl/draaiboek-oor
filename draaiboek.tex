\documentclass[a4paper,11pt]{article}

\usepackage[dutch]{babel}
\usepackage[utf8]{inputenc}
\usepackage[T1]{fontenc}
\usepackage[margin=2.54cm]{geometry}

% Package om kaders te tekenen rond tekst. Gebruik \begin{framed} en \end{framed} rond de tekst die je wil omkaderen.
\usepackage{framed}

\usepackage{hyperref}

\title{Draaiboek OOR}
\author{Overkoepelende OnderwijsRaad (OOR) \\ Scientica Leuven vzw (Scientica-OOR) \\ Facultair Overlegorgaan Faculteit Wetenschappen}
% Gebruik date voor het versienummer. Voor een herziening van een versie, gebruik <nummer>.<revisienummer> (bijvoorbeeld 2020.1 voor een eerste herziening).
\date{Versie 2020}

\begin{document}
	\maketitle
	
	% Zorgt ervoor de de rest van de tekst onderaan de pagina staat.
	\vspace*{\fill}
	
	\textbf{Waarschuwing:} Dit document is niet absoluut. Het is slechts een beschrijving van de werking van OOR op dit moment en de ideeën die er circuleren over hoe OOR zou kunnen werken. Gebruik dit document dus enkel ter inspiratie of als documentatie, niet als een bron die strikt gevolgd moet worden. De voornaamste reden voor het maken van dit draaiboek is om een toekomstig onervaren bestuur de mogelijkheid te geven begeleid te worden, zonder voormalige bestuurders op te moeten sporen.\newline
	
	\textbf{Opmerkingen?} Pas gerust zaken in dit document zelf aan. Het is publiekelijk beschikbaar op \hyperlink{https://github.com/ruben-vl/draaiboek-oor}{github}\footnote{Indien je het document niet elektronisch hebt en dus niet op de link kan klikken, kan je de repository op volgende url vinden: \texttt{https://github.com/ruben-vl/draaiboek-oor}} en aanpassingen kunnen via een pull request in de master branch toegevoegd worden. Om je te laten toevoegen als collaborator of voor andere zaken kan je mailen naar rubenvanlaer@mail.com.\newline

	% Voeg je naam toe wanneer je toevoegingen hebt gemaakt aan dit draaiboek.
	\textbf{Auteurs:} Ruben Van Laer
	
	\newpage
	
	\tableofcontents
	
	\newpage
	
	\section{Definitie OOR}
	
	\section{Doel OOR}
	
	\section{Statuten Scientica Leuven vzw}
	
	
	
	\section{Huishoudelijk Reglement OOR}
	
		\subsection{Aanpassingen}
	
	\section{Logo en Embleem}
	
	\section{Website OOR}
	
	\section{De begroting}
	\label{begroting}
	
	\section{Drukwerk}
	\label{drukwerk}
	
	\section{Textiel}
	
	\section{E-mailadressen}
	
	\section{OOR-overleg}
	
		\subsection{Uitnodiging}
	
		\subsection{Agenda}
		
		\subsection{Geledingen}
		
		\subsection{Stemrecht}
	
		\subsection{Aanwezigheden}
		\label{overleg-aanwezigheden}
		
		% TODO: Nagaan of het opnemen van aanwezigheden verplicht is en zo niet, dat best aan het HR laten toevoegen.
	
		De aanwezigheden tijdens het OOR-overleg worden opgenomen door de notulist. Deze dienen voornamelijk voor het verslag, zie \ref{verslag-aanwezigheden}. Deze opname zou enkel de namen van de aanwezigen kunnen bevatten, of meer informatie zoals de opleiding of het mandaat. In het werkingsjaar 2019-2020 werd er steeds een papier rondgedeeld met hoofding 'naam, opleiding, handtekening'. Een goede gewoonte in dat geval is om tijdens de rondvraag na te vragen of iedereen de aanwezigheidslijst ingevuld heeft. Dit kan zowel door de voorzitter of de notulist gebeuren. Indien je de aanwezigen schrijfwerk wil besparen, kan deze lijst ook al de namen van alle stuvers op voorhand bevatten, waarna ze slechts moeten handtekenen. Dit vraagt wel extra drukwerk, zie \ref{drukwerk}, wat dan meegenomen moet worden in de begroting, zie \ref{begroting}.
		
		\subsection{Pauze}
		
		Na 60 of 90 minuten wordt er best een pauze ingelast van 5 tot 10 minuten. Dit maakt een toiletbezoek en eventueel onderlinge discussie mogelijk zonder dat het het overleg stoort.
	
	\section{Verslagen}
	\label{verslag}
	
		\subsection{Aanwezigheden}
		\label{verslag-aanwezigheden}

		De aanwezigheden worden opgenomen tijdens het OOR-overleg, zie \ref{overleg-aanwezigheden}. Deze worden dan in het verslag toegevoegd. In het verslag kunnen enkel de namen opgesomd worden, of ze kunnen gegroepeerd worden per geleding binnen OOR. Ook wordt er best aangegeven wie de notulist en voorzitter waren, zodat met latere vragen of opmerkingen, één van deze bereikt kan worden.
	
		\subsection{Voorbereiding Stura AV}
	
	\section{OOR trakteert}
	
	\section{Scientica AV}
	
	\section{Faculteit Wetenschappen}
	
	\section{Studentenraad KU Leuven}
	\label{stura}
	
	\section{Campusbibliotheek Arenberg (CBA)}
	
	\section{Voormalige Bestuurders}
	
	\begin{thebibliography}{9}
		\bibitem{statutenscientica} 
		Statuten Scientica Leuven vzw
		\bibitem{hroor}
		Huishoudelijk Reglement OOR
	\end{thebibliography}
	
	
	
\end{document}